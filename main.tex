%\documentclass{article}
\documentclass[10pt]{extarticle}
\usepackage{graphicx} % Required for inserting images

\usepackage[english]{babel}
\usepackage[T1]{fontenc}
\usepackage{lmodern,mathrsfs}
\usepackage{xparse}
\usepackage[inline,shortlabels]{enumitem}
\setlist{topsep=2pt,itemsep=2pt,parsep=0pt,partopsep=0pt}
\usepackage[dvipsnames]{xcolor}
\usepackage[utf8]{inputenc}
%\usepackage[a4paper,top=0.5in,bottom=0.2in,left=0.5in,right=0.5in,footskip=0.3in,includefoot]{geometry}
\usepackage[most]{tcolorbox}
\tcbuselibrary{minted} % tcolorbox minted library, required to use the "minted" tcb listing engine (this library is not loaded by the option [most])
%\usepackage{minted} % Allows input of raw code, such as Python code
%\usepackage{caption}
\usepackage{minted}
\usemintedstyle{one-dark}
\setminted[python]{bgcolor=gray!40!black, linenos}
%\usemintedstyle{one-dark}{colback=gray!40!black,colframe=blue}
%\usemintedstyle{colorful}

%\usepackage{pythontex}
% https://chatgpt.com/c/cabd51ce-b71f-4919-8846-7ee554886474
\usepackage{xcolor}
\usepackage{tcolorbox}

\usepackage{catchfile}

%\usepackage{pythonhighlight}

\usepackage[colorlinks]{hyperref} % ALWAYS load this package LAST

% Custom tcolorbox style for Python code (not the code or the box it appears in, just the options for the box)
\tcbset{
    pythoncodebox/.style={
        enhanced jigsaw,breakable,
        colback=gray!10,colframe=gray!20!black,
        boxrule=1pt,top=2pt,bottom=2pt,left=2pt,right=2pt,
        sharp corners,before skip=10pt,after skip=10pt,
        attach boxed title to top left,
        boxed title style={empty,
            top=0pt,bottom=0pt,left=2pt,right=2pt,
            interior code={\fill[fill=tcbcolframe] (frame.south west)
                --([yshift=-4pt]frame.north west)
                to[out=90,in=180] ([xshift=4pt]frame.north west)
                --([xshift=-8pt]frame.north east)
                to[out=0,in=180] ([xshift=16pt]frame.south east)
                --cycle;
            }
        },
        title={#1}, % Argument of pythoncodebox specifies the title
        fonttitle=\sffamily\bfseries
    },
    pythoncodebox/.default={}, % Default is No title
    %%% Starred version has no frame %%%
    pythoncodebox*/.style={
        enhanced jigsaw,breakable,
        colback=gray!10,coltitle=gray!20!black,colbacktitle=tcbcolback,
        frame hidden,
        top=2pt,bottom=2pt,left=2pt,right=2pt,
        sharp corners,before skip=10pt,after skip=10pt,
        attach boxed title to top text left={yshift=-1mm},
        boxed title style={empty,
            top=0pt,bottom=0pt,left=2pt,right=2pt,
            interior code={\fill[fill=tcbcolback] (interior.south west)
                --([yshift=-4pt]interior.north west)
                to[out=90,in=180] ([xshift=4pt]interior.north west)
                --([xshift=-8pt]interior.north east)
                to[out=0,in=180] ([xshift=16pt]interior.south east)
                --cycle;
            }
        },
        title={#1}, % Argument of pythoncodebox specifies the title
        fonttitle=\sffamily\bfseries
    },
    pythoncodebox*/.default={}, % Default is No title
}

% Custom tcolorbox for Python code (not the code itself, just the box it appears in)
\newtcolorbox{pythonbox}[1][]{pythoncodebox=#1}
\newtcolorbox{pythonbox*}[1][]{pythoncodebox*=#1} % Starred version has no frame

% Custom minted environment for Python code, NOT using tcolorbox
\newminted{python}{autogobble,breaklines,mathescape}

% Custom tcblisting environment for Python code, using the "minted" tcb listing engine
% Adapted from https://tex.stackexchange.com/a/402096
\NewTCBListing{python}{ !O{} !D(){} !G{} }{
    listing engine=minted,
    listing only,
    pythoncodebox={#1}, % First argument specifies the title (if any)
    minted language=python,
    minted options/.expanded={
        autogobble,breaklines,mathescape,
        #2 % Second argument, delimited by (), denotes options for the minted environment
    },
    #3 % Third argument, delimited by {}, denotes options for the tcolorbox
}

%%% Starred version has no frame %%%
\NewTCBListing{python*}{ !O{} !D(){} !G{} }{
    listing engine=minted,
    listing only,
    pythoncodebox*={#1}, % First argument specifies the title (if any)
    minted language=python,
    minted options/.expanded={
        autogobble,breaklines,mathescape,
        #2 % Second argument, delimited by (), denotes options for the minted environment
    },
    #3 % Third argument, delimited by {}, denotes options for the tcolorbox
}

% verbbox environment, for showing verbatim text next to code output (for package documentation and user learning purposes)
\NewTCBListing{verbbox}{ !O{} }{
    listing engine=minted,
    minted language=latex,
    boxrule=1pt,sidebyside,skin=bicolor,
    colback=gray!10,colbacklower=white,valign=center,
    top=2pt,bottom=2pt,left=2pt,right=2pt,
    #1
} % Last argument allows more tcolorbox options to be added

\setlength{\parindent}{0.2in}
\setlength{\parskip}{0pt}
\setlength{\columnseprule}{0pt}

\makeatletter
% Redefining the title block
\renewcommand\maketitle{
    \null\vspace{4mm}
    \begin{center}
        {\Huge\sffamily\bfseries\selectfont\@title}\\
            \vspace{4mm}
        {\Large\sffamily\selectfont\@author}\\
            \vspace{4mm}
        {\large\sffamily\selectfont\@date}
    \end{center}
    \vspace{6mm}
}
% Adapted from https://tex.stackexchange.com/questions/483953/how-to-add-new-macros-like-author-without-editing-latex-ltx?noredirect=1&lq=1
\makeatother

\newmintedfile[pythoncode3]{python}{
bgcolor=mintedbackground,
fontfamily=tt,
linenos=true,
numberblanklines=true,
numbersep=5pt,
gobble=0,
frame=leftline,
framerule=0.4pt,
framesep=2mm,
funcnamehighlighting=true,
tabsize=4,
obeytabs=false,
mathescape=false
samepage=false, %with this setting you can force the list to appear on the same page
showspaces=false,
showtabs =false,
texcl=false
}

% Define a custom tcolorbox for Python code
\newtcblisting{pythoncodeA}[2][]{%
  colback=gray!40!black, % Background color
  colframe=blue,         % Frame color
  listing only,
  listing options={%
    language=python,     % Set the language to Python
    style=one-dark,      % Apply a Pygments style (if available)
  },
  title=#2,              % Title of the code block
  #1                     % Additional options (if any)
}

\newcommand{\py}[2]{\input{|python include.py '#1' '#2'}}

\title{Collatz Writing}
\author{Eivind Magnus Hvidevold}
\date{August 2024}

\begin{document}

\maketitle

\section{Introduction}

I want to calculate. The calculations are of arithmetic. I will shortly provide a list of arithmetical operations.
They are: "addition", "subtraction", "multiplication" and "division". I think I will choose the long form of the arithmetical operations, such as in \href{https://en.wikipedia.org/wiki/Long_division}{long division}.

I would like to extend the base arithmetic operations to any base. Furthermore, I would like to extend the base arithmetic operations to a more abstract numerical integer representation than any base. Perhaps it will be based on bucket multipliers, which are based on powers of the base that the digits are multiplied with.  An example of an integer is 1239, which means 1 * 1000 + 2 * 100 + 30 * 10 + 9 * 1.

This raises a question. How do I decide the range of representable integers, if it is not a simple base structure of multipliers. How do I make sure that all integers are representable? Well, it's trial and error with different numerical representations and abstractions over these while proceeding to look for a sufficiently general pattern of numerical integer representation to have an impact on the $3n + 1$ problem.

I choose python3 for my programming. How should I represent the integer numbers? Should I allow fractional numbers? If I don't already know whether I want to allow fractions, I should abstract over the decision and make a class that's either easily modifiable or extensible in some ways. I think it will be good in any case to abstract the maximum possible, while entertaining the reader to follow through from concrete examples to more abstract ideas of numerical integer and fraction representations.

I should refresh my mind on operator overloading in python3. That way I can think better and prepare about the subject matter at hand. The following document should suffice: \href{https://docs.python.org/3/reference/datamodel.html#special-method-names}{Python3 Special Method Names},
linked from \href{https://stackoverflow.com/questions/2400635/comprehensive-guide-to-operator-overloading-in-python}{Stack Overflow Question}. 3.3.8. Emulating numeric types is most relevant.

Is it sufficient to represent a number by a list of digits and an "potentially infinite" list of base multipliers? Two things need to be specified, the set of possible digits (and their numerical value) and a function from the integer position of the digit in a left-right string, such as "$1301324532$", where it is assumed that the rightmost digit is multiplied by $10^0 = 1$, the second rightmost digit by $3$ and the leftmost digit by $10^9$.

\input{|python -c 'print(1+2)'}

\input{|python test.py}

\begin{python}[How to calculate a factorial](style=one-dark){colback=gray!40!black,colframe=blue}
def factorial(n):
product=1 # Start with $1$
for k in range(2,n+1): # For each $k=2,3,\ldots,n$,
product*=k # Multiply by $k$
return product
\end{python}

\end{document}
